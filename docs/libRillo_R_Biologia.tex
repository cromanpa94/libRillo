% Options for packages loaded elsewhere
\PassOptionsToPackage{unicode}{hyperref}
\PassOptionsToPackage{hyphens}{url}
%
\documentclass[
]{book}
\usepackage{amsmath,amssymb}
\usepackage{lmodern}
\usepackage{iftex}
\ifPDFTeX
  \usepackage[T1]{fontenc}
  \usepackage[utf8]{inputenc}
  \usepackage{textcomp} % provide euro and other symbols
\else % if luatex or xetex
  \usepackage{unicode-math}
  \defaultfontfeatures{Scale=MatchLowercase}
  \defaultfontfeatures[\rmfamily]{Ligatures=TeX,Scale=1}
\fi
% Use upquote if available, for straight quotes in verbatim environments
\IfFileExists{upquote.sty}{\usepackage{upquote}}{}
\IfFileExists{microtype.sty}{% use microtype if available
  \usepackage[]{microtype}
  \UseMicrotypeSet[protrusion]{basicmath} % disable protrusion for tt fonts
}{}
\makeatletter
\@ifundefined{KOMAClassName}{% if non-KOMA class
  \IfFileExists{parskip.sty}{%
    \usepackage{parskip}
  }{% else
    \setlength{\parindent}{0pt}
    \setlength{\parskip}{6pt plus 2pt minus 1pt}}
}{% if KOMA class
  \KOMAoptions{parskip=half}}
\makeatother
\usepackage{xcolor}
\IfFileExists{xurl.sty}{\usepackage{xurl}}{} % add URL line breaks if available
\IfFileExists{bookmark.sty}{\usepackage{bookmark}}{\usepackage{hyperref}}
\hypersetup{
  pdftitle={LibRillo de R y Biologia},
  pdfauthor={Cristian Román Palacios},
  hidelinks,
  pdfcreator={LaTeX via pandoc}}
\urlstyle{same} % disable monospaced font for URLs
\usepackage[margin=1in]{geometry}
\usepackage{color}
\usepackage{fancyvrb}
\newcommand{\VerbBar}{|}
\newcommand{\VERB}{\Verb[commandchars=\\\{\}]}
\DefineVerbatimEnvironment{Highlighting}{Verbatim}{commandchars=\\\{\}}
% Add ',fontsize=\small' for more characters per line
\usepackage{framed}
\definecolor{shadecolor}{RGB}{248,248,248}
\newenvironment{Shaded}{\begin{snugshade}}{\end{snugshade}}
\newcommand{\AlertTok}[1]{\textcolor[rgb]{0.94,0.16,0.16}{#1}}
\newcommand{\AnnotationTok}[1]{\textcolor[rgb]{0.56,0.35,0.01}{\textbf{\textit{#1}}}}
\newcommand{\AttributeTok}[1]{\textcolor[rgb]{0.77,0.63,0.00}{#1}}
\newcommand{\BaseNTok}[1]{\textcolor[rgb]{0.00,0.00,0.81}{#1}}
\newcommand{\BuiltInTok}[1]{#1}
\newcommand{\CharTok}[1]{\textcolor[rgb]{0.31,0.60,0.02}{#1}}
\newcommand{\CommentTok}[1]{\textcolor[rgb]{0.56,0.35,0.01}{\textit{#1}}}
\newcommand{\CommentVarTok}[1]{\textcolor[rgb]{0.56,0.35,0.01}{\textbf{\textit{#1}}}}
\newcommand{\ConstantTok}[1]{\textcolor[rgb]{0.00,0.00,0.00}{#1}}
\newcommand{\ControlFlowTok}[1]{\textcolor[rgb]{0.13,0.29,0.53}{\textbf{#1}}}
\newcommand{\DataTypeTok}[1]{\textcolor[rgb]{0.13,0.29,0.53}{#1}}
\newcommand{\DecValTok}[1]{\textcolor[rgb]{0.00,0.00,0.81}{#1}}
\newcommand{\DocumentationTok}[1]{\textcolor[rgb]{0.56,0.35,0.01}{\textbf{\textit{#1}}}}
\newcommand{\ErrorTok}[1]{\textcolor[rgb]{0.64,0.00,0.00}{\textbf{#1}}}
\newcommand{\ExtensionTok}[1]{#1}
\newcommand{\FloatTok}[1]{\textcolor[rgb]{0.00,0.00,0.81}{#1}}
\newcommand{\FunctionTok}[1]{\textcolor[rgb]{0.00,0.00,0.00}{#1}}
\newcommand{\ImportTok}[1]{#1}
\newcommand{\InformationTok}[1]{\textcolor[rgb]{0.56,0.35,0.01}{\textbf{\textit{#1}}}}
\newcommand{\KeywordTok}[1]{\textcolor[rgb]{0.13,0.29,0.53}{\textbf{#1}}}
\newcommand{\NormalTok}[1]{#1}
\newcommand{\OperatorTok}[1]{\textcolor[rgb]{0.81,0.36,0.00}{\textbf{#1}}}
\newcommand{\OtherTok}[1]{\textcolor[rgb]{0.56,0.35,0.01}{#1}}
\newcommand{\PreprocessorTok}[1]{\textcolor[rgb]{0.56,0.35,0.01}{\textit{#1}}}
\newcommand{\RegionMarkerTok}[1]{#1}
\newcommand{\SpecialCharTok}[1]{\textcolor[rgb]{0.00,0.00,0.00}{#1}}
\newcommand{\SpecialStringTok}[1]{\textcolor[rgb]{0.31,0.60,0.02}{#1}}
\newcommand{\StringTok}[1]{\textcolor[rgb]{0.31,0.60,0.02}{#1}}
\newcommand{\VariableTok}[1]{\textcolor[rgb]{0.00,0.00,0.00}{#1}}
\newcommand{\VerbatimStringTok}[1]{\textcolor[rgb]{0.31,0.60,0.02}{#1}}
\newcommand{\WarningTok}[1]{\textcolor[rgb]{0.56,0.35,0.01}{\textbf{\textit{#1}}}}
\usepackage{longtable,booktabs,array}
\usepackage{calc} % for calculating minipage widths
% Correct order of tables after \paragraph or \subparagraph
\usepackage{etoolbox}
\makeatletter
\patchcmd\longtable{\par}{\if@noskipsec\mbox{}\fi\par}{}{}
\makeatother
% Allow footnotes in longtable head/foot
\IfFileExists{footnotehyper.sty}{\usepackage{footnotehyper}}{\usepackage{footnote}}
\makesavenoteenv{longtable}
\usepackage{graphicx}
\makeatletter
\def\maxwidth{\ifdim\Gin@nat@width>\linewidth\linewidth\else\Gin@nat@width\fi}
\def\maxheight{\ifdim\Gin@nat@height>\textheight\textheight\else\Gin@nat@height\fi}
\makeatother
% Scale images if necessary, so that they will not overflow the page
% margins by default, and it is still possible to overwrite the defaults
% using explicit options in \includegraphics[width, height, ...]{}
\setkeys{Gin}{width=\maxwidth,height=\maxheight,keepaspectratio}
% Set default figure placement to htbp
\makeatletter
\def\fps@figure{htbp}
\makeatother
\setlength{\emergencystretch}{3em} % prevent overfull lines
\providecommand{\tightlist}{%
  \setlength{\itemsep}{0pt}\setlength{\parskip}{0pt}}
\setcounter{secnumdepth}{5}
\ifLuaTeX
  \usepackage{selnolig}  % disable illegal ligatures
\fi
\usepackage[]{natbib}
\bibliographystyle{plainnat}

\title{LibRillo de R y Biologia}
\author{Cristian Román Palacios}
\date{2022-06-19}

\begin{document}
\maketitle

{
\setcounter{tocdepth}{2}
\tableofcontents
}
\hypertarget{pruxf3logo}{%
\chapter*{Prólogo}\label{pruxf3logo}}


Este libro es una compilacion para consulta rapida sobre diferentes aspectos que combinan el uso de \texttt{R} con diferentes temas de biologia. El \texttt{LibRillo\ de\ R\ y\ Biologia} pretende servir de referencia sobre temas particulares, enfocando capitulos sobre objetivos especificos y desarrollando brevemente el contenido principalmente alrededor de un parquete en \texttt{R}. Este librillo cubre temas que engloban desde lo mas fundamental en \texttt{R} asi como aplicaciones que combinan \texttt{R}, \texttt{git}, \texttt{GitHub}, entre otros.

\hypertarget{estructura-del-libro}{%
\section*{Estructura del libro}\label{estructura-del-libro}}


Este \texttt{LibRillo\ de\ R\ y\ Biologia} pretende avanzar a quien lo consulte en temas especificos que relacionan diferentes ambitos y aspectos biologicos con \texttt{R}. Sin embargo, el libro no se enfoca solamente en revisar la aplicacion de conceptus usando tecnicas. En el capitulo 1 XXX\ldots{}

\hypertarget{convenciones-generales-del-libro}{%
\section*{Convenciones generales del libro}\label{convenciones-generales-del-libro}}


Este libro fue generado usando principalmente \texttt{R} \texttt{knitr} y \texttt{bookdown}. En general, los bloques de codigo que se incluyen en el libro permiten una ejecucion sencilla e intuitiva de los temas. En diferentes secciones del libro, se pretende generar salidad graficas y numericas directamente usando codigos en \texttt{R}. Al inicio de cada capitulo incluyo informacion sobre los paquetes que son relevantes para llevar a cabo los diferentes objetivos del mismo.

La informacion contenida en este libro fue compilada usando la siguiente sesion de \texttt{R}:

\begin{Shaded}
\begin{Highlighting}[]
\FunctionTok{sessionInfo}\NormalTok{()}
\end{Highlighting}
\end{Shaded}

\begin{verbatim}
## R version 4.2.0 (2022-04-22)
## Platform: x86_64-apple-darwin17.0 (64-bit)
## Running under: macOS Catalina 10.15.7
## 
## Matrix products: default
## BLAS:   /Library/Frameworks/R.framework/Versions/4.2/Resources/lib/libRblas.0.dylib
## LAPACK: /Library/Frameworks/R.framework/Versions/4.2/Resources/lib/libRlapack.dylib
## 
## locale:
## [1] en_US.UTF-8/en_US.UTF-8/en_US.UTF-8/C/en_US.UTF-8/en_US.UTF-8
## 
## attached base packages:
## [1] stats     graphics  grDevices utils     datasets  methods   base     
## 
## loaded via a namespace (and not attached):
##  [1] compiler_4.2.0  magrittr_2.0.3  fastmap_1.1.0   bookdown_0.27  
##  [5] cli_3.3.0       htmltools_0.5.2 tools_4.2.0     rstudioapi_0.13
##  [9] yaml_2.3.5      stringi_1.7.6   rmarkdown_2.14  knitr_1.39     
## [13] stringr_1.4.0   digest_0.6.29   xfun_0.31       rlang_1.0.2    
## [17] evaluate_0.15
\end{verbatim}

Pretendo no agregar simbolos como \texttt{\textgreater{}} y \texttt{+} en el codigo de \texttt{R} dentro de este libro. En general, los codigos aparecen destacados en cajas grises:

\begin{Shaded}
\begin{Highlighting}[]
\NormalTok{a }\OtherTok{\textless{}{-}} \FunctionTok{c}\NormalTok{(}\DecValTok{2}\NormalTok{, }\DecValTok{4}\NormalTok{, }\DecValTok{5}\NormalTok{)}
\end{Highlighting}
\end{Shaded}

Las salidas de codigo tambien aparecen en un bloque gris. Sin embargo, las lineas de salida incluyen los caracteres \texttt{\#\#} antes del contenido de cada linea. Por ejemplo, multipliquemos el vector \texttt{a} creado anteriormente por \texttt{20}:

\begin{verbatim}
## [1]  40  80 100
\end{verbatim}

La salida de esta operacion necesariamente tiene los \texttt{\#\#} en el bloque. Por lo tanto, unicamente el codigo que \textbf{no} tiene \texttt{\#\#} debe ser copiado (en caso de ser necesario). Las salidas de codigo realmente no son ran relevantes para reproducir el codigo.

Por ultimo, el nombre de paquetes se indica en negrilla dentro del texto (e.g.~\textbf{bookdown}), las functiones se presentan seguidas de parentesis (e.g.~\texttt{render\_book()}), los dobles dos puntos (\texttt{::}) significan que un determinado objeto pertenece a un paquete determinado (e.g.~\texttt{bookdown::render\_book()}), y codigos en el texto aparecen en formato de maquina de escribir (e.g.~\texttt{R}).

Estas son las convenciones mas generales del libro. Otras convenciones mas particulares y relevantes a otros temas seran discutidas en los capitulos siguientes.

\hypertarget{dedicacion}{%
\section*{Dedicacion}\label{dedicacion}}


Este libro esta escrito principalmente para estudiantes de universidades publicas o pertenecientes a grupos historicamente excluidos de la academia. Escribo el libro con mucho cariño, pensando en que ayudara a equilibrar la balanza entre quienes tienen acceso directo a recursos ``avanzados'' en ingles y aquellos que por diferentes razones, solamente acceden a informacion en español. El libro es de acceso abierto y no pretendo cobrar por su uso.

\hypertarget{agradecimientos}{%
\section*{Agradecimientos}\label{agradecimientos}}


Mis agradecimientos mas fuertes van para mi padre y madre, Cesar Roman y Yocasta Palacios. Al parecer, de mi padre, Ictiologo y profesor titular en la Universidad del Quindio, termine heredando la profesion. De mi madre, profesora de preescolar en diferentes instituciones del Quindio (Colombia), la pasion y paciencia por enseñar y ayudar. Grandes agradecimientos van tambien a mi hermano, Carlos, ingeniero en sistemas, gran programador y conversador, quien por su cercania me mostro que escribir codigo (al menos al nivel basico que yo hago), no es imposible. Este libro lo empiezo a escribir estado en Tucson, AZ. Gracias a mi pareja, Heidi E. Steiner, quien me apoya con su actitud, tiempo, y cariño, mientras me dedico por raticos a escribir apartados de este libro. A estos cuatro personajes, motor de mi vida, todo mi cariño y admiracion.

\hypertarget{sobre-el-autor}{%
\chapter*{Sobre el autor}\label{sobre-el-autor}}


Cristian Román Palacios es Profesor Asistente en la Universidad de Arizona desde el 2021. Cristian es Biologo de la Universidad del Valle (2015), Magister y Ph.D.~en Ecologia y Evolucion (2019, 2020) de la Universidad de Arizona, con cargos postdoctorales en Temple University (2021) y University of California, Los Angeles (2021). Los interestes investigativos de Cristian contemplan temas de evolucion a grandes escalas taxonomicas, asi como la relacion entre cambio climatico y diversidad. Sin embargo, el aprendizaje de areas relacionadas con programacion resulto ser una prioridad durante su formacion academica. Ahora pretende divulgar de forma sencilla, contenidos de \texttt{R} y biologia usando diferentes recursos, uno de estos son los libros. Actualmente, Cristian enseña cursos de Machine Learning y Data Mining en la Escuela de La Informacion, en la Universidad de Arizona. Mas informacion sobre el autor esta disponible en su \href{https://cromanpa94.github.io/cromanpa/}{pagina web}.

\hypertarget{una-breve-introduccion-a-r}{%
\chapter{\texorpdfstring{Una breve introduccion a \texttt{R}}{Una breve introduccion a R}}\label{una-breve-introduccion-a-r}}

\texttt{R} será el principal lenguaje que guiara el libro. En diferentes disciplinas biologicas, el uso de \texttt{R} se ha extendido significativamente durante la ultima decada. Sin embargo, \texttt{R}, por ser un lenguaje principalmente enfocado en aspectos estadisticos, tiene ciertas limitaciones en cuanto a velocidad y flexibilidad. Por otro lado, debido a la compartimentalizacion de implementaciones metodologicas en codigo, aprender y usar \texttt{R} es probablemente central en un campo biologico determinado pero no en otros. Otros recursos pueden ser usados para seguir la estructura del presente libro en otros lenguajes (e.g.~\texttt{python}). Por ahora, nos enfocaremos en como instalar y usar \texttt{R}. Por simplicidad, vamos a procurar trabajar desde \texttt{RStudio}.

\hypertarget{r-y-rstudio}{%
\section{\texorpdfstring{\texttt{R} y \texttt{RStudio}}{R y RStudio}}\label{r-y-rstudio}}

Aunque no son equivalentes, en diferentes contextos, \texttt{R} y \texttt{RStudio} se usan erroneamente de forma intercambiable. \texttt{R} es explicitamente un lenguaje de programacion, desarrollado por decadas como parte del \texttt{R\ Core\ Team} y la \texttt{R\ Foundation\ for\ Statistical\ Computing}. En pocas palabras, \texttt{R} es un lenguaje en si mismo, que esta enfocado en computaciones estadisticas. \texttt{RStudio} requiere de \texttt{R} para functionar. Especificamente, \texttt{RStudio} permite manejar (potentialmente) mas eficientemente y esteticamente el espacio de trabajo en \texttt{R}.

\hypertarget{instalando-r-y-rstudio}{%
\subsection{\texorpdfstring{Instalando \texttt{R} y \texttt{RStudio}}{Instalando R y RStudio}}\label{instalando-r-y-rstudio}}

Dependiendo del sistema operativo en consideracion, existen multiples maneras de instalar \texttt{R}. En MacOS, vamos a usar \texttt{homebrew} para instalar paquetes y demas software que necesitemos. Para instalar \texttt{homebrew}, pueden correr la siguiente linea en la linea de comandos:

\begin{Shaded}
\begin{Highlighting}[]
\ExtensionTok{/bin/bash} \AttributeTok{{-}c} \StringTok{"}\VariableTok{$(}\ExtensionTok{curl} \AttributeTok{{-}fsSL}\NormalTok{ https://raw.githubusercontent.com/Homebrew/install/HEAD/install.sh}\VariableTok{)}\StringTok{"} 
\end{Highlighting}
\end{Shaded}

Si estan usando linux, intentaremos hacer las instalaciones usando \texttt{apt\ install} para instalar programas desde la linea de comandos. En Windows, los ejecutables estan generalente disponibles para los programas que estaremos usando.

\hypertarget{instalando-r}{%
\subsubsection{\texorpdfstring{Instalando \texttt{R}}{Instalando R}}\label{instalando-r}}

En linux, la instalacion de \texttt{R} normalente requiere de un par de comandos:

\begin{Shaded}
\begin{Highlighting}[]
\FunctionTok{sudo}\NormalTok{ apt update}
\FunctionTok{sudo}\NormalTok{ apt install r{-}base}
\end{Highlighting}
\end{Shaded}

En MacOs y desde \texttt{homebrew}, la instalacion de \texttt{R} involucra la siguiente linea:

\begin{Shaded}
\begin{Highlighting}[]
\ExtensionTok{brew}\NormalTok{ install r}
\end{Highlighting}
\end{Shaded}

En Linux y MacOs, la instalacion debe crear una nueva aplicacion. Sin embargo, desde la misma linea de comandos, correr

\begin{Shaded}
\begin{Highlighting}[]
\ExtensionTok{R}
\end{Highlighting}
\end{Shaded}

debe permitir ingresar a \texttt{R} directamente. Por ultimo, es potentialmente preferible descagar e instalar \texttt{R} usando el instalador en la pagina oficial de \texttt{R} project usando este \href{https://cran.r-project.org/bin/windows/base/}{link}.

\hypertarget{instalando-rstudio}{%
\subsubsection{\texorpdfstring{Instalando \texttt{RStudio}}{Instalando RStudio}}\label{instalando-rstudio}}

Instalar \texttt{RStudio} no es totalmente necesario para seguir las instrucciones en todos los capitulos de este librillo. Sin embargo, es ultil para muchas de ellas. Por otro lado, \texttt{RStudio} no es completamente necesario para correr \texttt{R}. Para descargarlo e instalarlo pueden seguir las instrucciones en la \href{https://www.rstudio.com/products/rstudio/}{pagina oficial} de \texttt{RStudio}.

\texttt{RStudio} tambien puede ser instalado desde la linea de comando. En \href{https://docs.rstudio.com/rpm/installation/}{Ubuntu Linux}, instalar \texttt{RStudio} toma las siguientes lineas:

\begin{Shaded}
\begin{Highlighting}[]
\FunctionTok{sudo}\NormalTok{ apt update}
\FunctionTok{sudo}\NormalTok{ apt install gdebi{-}core}
\FunctionTok{wget}\NormalTok{ https://cdn.rstudio.com/package{-}manager/ubuntu/amd64/rstudio{-}pm\_2022.04.0{-}7\_amd64.deb}
\FunctionTok{sudo}\NormalTok{ gdebi rstudio{-}pm\_2022.04.0{-}7\_amd64.deb}
\end{Highlighting}
\end{Shaded}

En MacOS, \texttt{homebrew} puede usarse para instalar \texttt{RStudio} muy facilmente:

\begin{Shaded}
\begin{Highlighting}[]
\ExtensionTok{brew}\NormalTok{ install }\AttributeTok{{-}{-}cask}\NormalTok{ rstudio}
\end{Highlighting}
\end{Shaded}

En Windows, la recomendacion es descargar la aplicacion directamente desde la pagina de \texttt{RStudio} en el siguiente \href{https://www.rstudio.com/products/rstudio/}{enlace}.

\hypertarget{primeros-pinitos-en-r}{%
\section{\texorpdfstring{Primeros pinitos en \texttt{R}}{Primeros pinitos en R}}\label{primeros-pinitos-en-r}}

Ahora que tenemos \texttt{R} instalado vamos a empezar a conocer algunos aspectos basicos del lenguaje!

\hypertarget{exportar-datos}{%
\subsection{Exportar datos}\label{exportar-datos}}

\hypertarget{r-un-poco-mas-avanzado}{%
\section{R un poco mas avanzado\ldots{}}\label{r-un-poco-mas-avanzado}}

\hypertarget{for-loops}{%
\subsection{\texorpdfstring{\texttt{for} loops}{for loops}}\label{for-loops}}

\hypertarget{vectorizando}{%
\subsection{Vectorizando}\label{vectorizando}}

\hypertarget{functiones-de-la-familia-apply}{%
\subsection{\texorpdfstring{Functiones de la familia *\texttt{apply}}{Functiones de la familia *apply}}\label{functiones-de-la-familia-apply}}

\hypertarget{data.table}{%
\subsection{\texorpdfstring{\texttt{data.table}}{data.table}}\label{data.table}}

\hypertarget{el-tidyverso}{%
\subsection{\texorpdfstring{El \texttt{tidy}verso}{El tidyverso}}\label{el-tidyverso}}

\hypertarget{personalizando-r}{%
\subsection{Personalizando R}\label{personalizando-r}}

Mensaje de inicio

\hypertarget{paginas-web-con-github-rstudio-y-r}{%
\chapter{\texorpdfstring{Paginas web con \texttt{GitHub}, \texttt{RStudio}, y \texttt{R}}{Paginas web con GitHub, RStudio, y R}}\label{paginas-web-con-github-rstudio-y-r}}

\hypertarget{jekyll}{%
\section{Jekyll}\label{jekyll}}

\hypertarget{hugo}{%
\section{Hugo}\label{hugo}}

\hypertarget{usando-github-rstudio-y-r}{%
\chapter{\texorpdfstring{Usando \texttt{GitHub}, \texttt{RStudio}, y \texttt{R}}{Usando GitHub, RStudio, y R}}\label{usando-github-rstudio-y-r}}

\hypertarget{git}{%
\section{\texorpdfstring{\texttt{Git}}{Git}}\label{git}}

\hypertarget{github}{%
\section{\texorpdfstring{\texttt{GitHub}}{GitHub}}\label{github}}

\hypertarget{rstudio-e-integracion-con-github}{%
\section{\texorpdfstring{\texttt{RStudio} e integracion con \texttt{GitHub}}{RStudio e integracion con GitHub}}\label{rstudio-e-integracion-con-github}}

\hypertarget{regresiones-lineales}{%
\chapter{Regresiones lineales}\label{regresiones-lineales}}

\hypertarget{regresiones-mas-avanzadas}{%
\chapter{Regresiones mas avanzadas}\label{regresiones-mas-avanzadas}}

\hypertarget{regresiones-bayesianas}{%
\section{Regresiones Bayesianas}\label{regresiones-bayesianas}}

\hypertarget{ecologia-basica}{%
\chapter{Ecologia basica}\label{ecologia-basica}}

\hypertarget{indices-de-diversidad}{%
\section{Indices de diversidad}\label{indices-de-diversidad}}

\hypertarget{curvas-de-acumulacion}{%
\section{Curvas de acumulacion}\label{curvas-de-acumulacion}}

\hypertarget{rarefaccion}{%
\section{Rarefaccion}\label{rarefaccion}}

\hypertarget{usando-datos-del-gbif}{%
\section{Usando datos del gbif}\label{usando-datos-del-gbif}}

\hypertarget{modelos-de-distribucion}{%
\section{Modelos de distribucion}\label{modelos-de-distribucion}}

\hypertarget{filogenetica-basica-en-r}{%
\chapter{\texorpdfstring{Filogenetica basica en \texttt{R}}{Filogenetica basica en R}}\label{filogenetica-basica-en-r}}

\hypertarget{descargar-secuencias-de-genbank}{%
\section{\texorpdfstring{Descargar secuencias de \texttt{GenBank}}{Descargar secuencias de GenBank}}\label{descargar-secuencias-de-genbank}}

\hypertarget{usar-secuencias-existentes}{%
\section{Usar secuencias existentes}\label{usar-secuencias-existentes}}

\hypertarget{alinear-secuencias}{%
\section{Alinear secuencias}\label{alinear-secuencias}}

\hypertarget{visualizar-alineamientos}{%
\section{Visualizar alineamientos}\label{visualizar-alineamientos}}

\hypertarget{correr-arboles-filogeneticos}{%
\section{Correr arboles filogeneticos}\label{correr-arboles-filogeneticos}}

\hypertarget{ips}{%
\section{IPS}\label{ips}}

\hypertarget{exportar-datos-desde-r}{%
\section{Exportar datos desde R}\label{exportar-datos-desde-r}}

\hypertarget{macroevolucion-basica-en-r}{%
\chapter{\texorpdfstring{Macroevolucion basica en \texttt{R}}{Macroevolucion basica en R}}\label{macroevolucion-basica-en-r}}

\hypertarget{regresiones-filogeneticas}{%
\section{Regresiones filogeneticas}\label{regresiones-filogeneticas}}

\hypertarget{estimaciones-de-seuxf1al-filogenetica}{%
\section{Estimaciones de señal filogenetica}\label{estimaciones-de-seuxf1al-filogenetica}}

\hypertarget{modelos-sse}{%
\section{Modelos SSE}\label{modelos-sse}}

\hypertarget{web-scraping-con-r}{%
\chapter{\texorpdfstring{Web scraping con \texttt{R}}{Web scraping con R}}\label{web-scraping-con-r}}

\hypertarget{machine-learning-con-r}{%
\chapter{\texorpdfstring{Machine learning con \texttt{R}}{Machine learning con R}}\label{machine-learning-con-r}}

\hypertarget{keras-en-r}{%
\section{\texorpdfstring{\texttt{Keras} en \texttt{R}}{Keras en R}}\label{keras-en-r}}

\hypertarget{tensorflow-en-r}{%
\section{\texorpdfstring{\texttt{Tensorflow} en \texttt{R}}{Tensorflow en R}}\label{tensorflow-en-r}}

\hypertarget{spark-r}{%
\section{\texorpdfstring{\texttt{Spark} \texttt{R}}{Spark R}}\label{spark-r}}

\hypertarget{data-mining-con-r}{%
\chapter{\texorpdfstring{Data Mining con \texttt{R}}{Data Mining con R}}\label{data-mining-con-r}}

\hypertarget{tests-estadisticos}{%
\chapter{Tests estadisticos}\label{tests-estadisticos}}

\hypertarget{estadistica-descriptiva-en-r}{%
\chapter{\texorpdfstring{Estadistica descriptiva en \texttt{R}}{Estadistica descriptiva en R}}\label{estadistica-descriptiva-en-r}}

\hypertarget{libros-paquetes-apps-en-r}{%
\chapter{\texorpdfstring{Libros, paquetes, apps en \texttt{R}}{Libros, paquetes, apps en R}}\label{libros-paquetes-apps-en-r}}

\hypertarget{libros-en-r}{%
\section{\texorpdfstring{Libros en \texttt{R}}{Libros en R}}\label{libros-en-r}}

\hypertarget{paquetes-en-r}{%
\section{\texorpdfstring{Paquetes en \texttt{R}}{Paquetes en R}}\label{paquetes-en-r}}

\hypertarget{shiny-apps-en-r}{%
\section{\texorpdfstring{\texttt{Shiny} apps en \texttt{R}}{Shiny apps en R}}\label{shiny-apps-en-r}}

\hypertarget{introduccion-a-rmarkdown}{%
\chapter{\texorpdfstring{Introduccion a \texttt{RMarkdown}}{Introduccion a RMarkdown}}\label{introduccion-a-rmarkdown}}

\hypertarget{autocalificar-tareas-en-r}{%
\chapter{\texorpdfstring{Autocalificar tareas en \texttt{R}}{Autocalificar tareas en R}}\label{autocalificar-tareas-en-r}}

\hypertarget{sha}{%
\section{SHA}\label{sha}}

\hypertarget{objetos-invisibles-en-el-espacio-de-trabajo}{%
\section{Objetos invisibles en el espacio de trabajo}\label{objetos-invisibles-en-el-espacio-de-trabajo}}

\hypertarget{graficos-en-r}{%
\chapter{\texorpdfstring{Graficos en \texttt{R}}{Graficos en R}}\label{graficos-en-r}}

\hypertarget{graficando-con-r-base}{%
\section{\texorpdfstring{Graficando con \texttt{R} \texttt{base}}{Graficando con R base}}\label{graficando-con-r-base}}

\hypertarget{graficando-con-ggplot2-en-r}{%
\section{\texorpdfstring{Graficando con \texttt{ggplot2} en \texttt{R}}{Graficando con ggplot2 en R}}\label{graficando-con-ggplot2-en-r}}

\hypertarget{macroevolucion-basica-en-r-1}{%
\chapter{\texorpdfstring{Macroevolucion basica en \texttt{R}}{Macroevolucion basica en R}}\label{macroevolucion-basica-en-r-1}}

\hypertarget{regresiones-filogeneticas-1}{%
\section{Regresiones filogeneticas}\label{regresiones-filogeneticas-1}}

\hypertarget{estimaciones-de-seuxf1al-filogenetica-1}{%
\section{Estimaciones de señal filogenetica}\label{estimaciones-de-seuxf1al-filogenetica-1}}

\hypertarget{modelos-sse-1}{%
\section{Modelos SSE}\label{modelos-sse-1}}

\hypertarget{manipulando-datos-en-r}{%
\chapter{\texorpdfstring{Manipulando datos en \texttt{R}}{Manipulando datos en R}}\label{manipulando-datos-en-r}}

\hypertarget{bases-de-datos-en-r}{%
\chapter{\texorpdfstring{Bases de datos en \texttt{R}}{Bases de datos en R}}\label{bases-de-datos-en-r}}

\hypertarget{sql}{%
\section{\texorpdfstring{\texttt{SQL}}{SQL}}\label{sql}}

\hypertarget{organizando-proyectos-reproducibles}{%
\chapter{Organizando proyectos reproducibles}\label{organizando-proyectos-reproducibles}}

\hypertarget{rstudio-projects}{%
\section{\texorpdfstring{\texttt{RStudio} projects}{RStudio projects}}\label{rstudio-projects}}

\hypertarget{github-repo}{%
\section{\texorpdfstring{\texttt{GitHub} repo}{GitHub repo}}\label{github-repo}}

\hypertarget{mejores-practicas-r}{%
\chapter{Mejores Practicas R}\label{mejores-practicas-r}}

\hypertarget{como-pedir-ayuda-de-forma-eficiente}{%
\chapter{Como pedir ayuda de forma eficiente?}\label{como-pedir-ayuda-de-forma-eficiente}}

\hypertarget{ejemplos-reproducibles}{%
\section{Ejemplos reproducibles}\label{ejemplos-reproducibles}}

\hypertarget{fuentes-potenciales}{%
\section{Fuentes potenciales}\label{fuentes-potenciales}}

\hypertarget{como-hacer-mapas-en-r}{%
\chapter{\texorpdfstring{Como hacer mapas en \texttt{R}}{Como hacer mapas en R}}\label{como-hacer-mapas-en-r}}

\hypertarget{base-r}{%
\section{\texorpdfstring{\texttt{Base} \texttt{R}}{Base R}}\label{base-r}}

\hypertarget{ggplot}{%
\section{\texorpdfstring{\texttt{ggplot}}{ggplot}}\label{ggplot}}

\hypertarget{simulando-datos}{%
\chapter{Simulando datos}\label{simulando-datos}}

\hypertarget{reportes-reproducibles}{%
\chapter{Reportes reproducibles}\label{reportes-reproducibles}}

\end{document}
